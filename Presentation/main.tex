\documentclass[aspectratio=169]{RUCPresentation}

%\usepackage{pgfforeach}
% \usepackage{foreach}

\usepackage{MyMath}
% \usepackage{animate}

\usepackage{siunitx}

\usepackage{tikz}
\usetikzlibrary{shapes, arrows, positioning, calc}
\usetikzlibrary{overlay-beamer-styles}


\def\k{\raisebox{1ex}{\rotatebox{180}{k}}}
% \newsavebox\tempboxbeamer

% \usetheme{Warsaw}
\usetheme{RUCTHEME}


% Defining a new coordinate system for the page:
%
% --------------------------
% |(-1,1)    (0,1)    (1,1)|
% |                        |
% |(-1,0)    (0,0)    (1,0)|
% |                        |
% |(-1,-1)   (0,-1)  (1,-1)|
% --------------------------

% For beamer title
\newcommand\makebeamertitle{\frame{\maketitle}}%

% For sections?
\graphicspath{{./images/}{../Thesis/images/phase/}{../Thesis/images/ode/}{../Thesis/images/markov/}}


\title{%
    Applying Fokker-Planck to analyze bacterial populations exhibiting hetero-resistence properties%
}
\title{%
    Understanding statistical distributions hetero-resistant bacterial populations through%
    fokker-planck or generalized statistical manifold%
}
\subtitle{  Mathematical\ \ Bioscience MSc }
% \newline supervisor: Morten Anderson}

\author{Anakin}

%\mode<presentation>

\begin{document}


\makebeamertitle


\section{Motivation}


\begin{frame}{Heteroresistence}

    \begin{columns}

        \column{0.5\framewidth}
        \vspace{4em}

        \begin{itemize}
            \item Bacteria can exhibit antibiotic resistences
            \item The mechanism is not as black and white as it seems
            \item High fitness cost as trait increases in quantity
            \item Can avoid bacterial anti-biotic resistance screening
        \end{itemize}


        \column{0.5\framewidth}
        
    \end{columns}

    \begin{tikzpicture}[overlay, remember picture]
        \coordinate (s) at (frame cs:0.1,0.5);
        \coordinate (e) at (frame cs:0.55,0.5);

        \draw (s) to (e);

        \path (s) to node[pos=0.3, sloped] (h1) {} (e);
        \path (s) to node[pos=0.7, sloped] (h2) {} (e);

        \path (s) to node[pos=0.1, sloped] (t11) {} (e);
        \path (s) to node[pos=0.25, sloped] (t21) {} (e);

        \path (s) to node[pos=0.9, sloped] (t12) {} (e);
        \path (s) to node[pos=0.75, sloped] (t22) {} (e);


        \draw[line width = 6, color=red] (h1) to (h2);

        \draw[line width = 4] (t11) to (t21);
        \draw[line width = 4] (t12) to (t22);

        \node[anchor = south east] at (s) {5'};
        \node[anchor = south west] at (e) {3'};

    \end{tikzpicture}

    \begin{tikzpicture}[overlay, remember picture]
        \coordinate (s) at (frame cs:0.1,0);
        \coordinate (e) at (frame cs:0.55,0);
        \coordinate (s2) at (frame cs:0.40,0);
        \coordinate (e2) at (frame cs:0.85,0);

        \draw (s) to (e);
        \draw (e) to (e2);

        \path (s) to node[pos=0.3, sloped] (h1) {} (e);
        \path (s) to node[pos=0.7, sloped] (h2) {} (e);


        \path (s) to node[pos=0.1, sloped] (t11) {} (e);
        \path (s) to node[pos=0.25, sloped] (t21) {} (e);

        \path (s) to node[pos=0.9, sloped] (t12) {} (e);
        \path (s) to node[pos=0.75, sloped] (t22) {} (e);


        \draw[line width = 6, color=red] (h1) to (h2);

        \draw[line width = 4] (t11) to (t21);
        \draw[line width = 4] (t12) to (t22);


        \path (s2) to node[pos=0.3, sloped] (h3) {} (e2);
        \path (s2) to node[pos=0.7, sloped] (h4) {} (e2);

        \path (s2) to node[pos=0.9, sloped] (t13) {} (e2);
        \path (s2) to node[pos=0.75, sloped] (t23) {} (e2);


        \draw[line width = 6, color=red] (h3) to (h4);

        \draw[line width = 4] (t13) to (t23);

        \node[anchor = south east] at (s) {5'};
        \node[anchor = south west] at (e2) {3'};

    \end{tikzpicture}

    \begin{tikzpicture}[overlay, remember picture]
        \coordinate (s) at (frame cs:0.1,-0.5);
        \coordinate (e) at (frame cs:0.55,-0.5);

        \draw (s) to (e);

        \path (s) to node[pos=0.3, sloped] (h1) {} (e);
        \path (s) to node[pos=0.7, sloped] (h2) {} (e);

        \path (s) to node[pos=0.1, sloped] (t11) {} (e);
        \path (s) to node[pos=0.25, sloped] (t21) {} (e);

        \path (s) to node[pos=0.9, sloped] (t12) {} (e);
        \path (s) to node[pos=0.75, sloped] (t22) {} (e);


        \draw[line width = 6, color=red] (h1) to (h2);

        \draw[line width = 4] (t11) to (t21);
        \draw[line width = 4] (t12) to (t22);

        \begin{scope}[node distance=5mm]
        \node (p1) [above=of t11] {};
        \node (p2) [below=of t11] {};

        \node (p3) [above=of t12] {};
        \node (p4) [below=of t12] {};

        \end{scope}

        \begin{scope}[node distance=0mm]
            \node [right= of p4] {n};
        \end{scope}

        \draw (p1) to[bend right] (p2);
        \draw (p3) to[bend left] (p4);


        \node[anchor = south east] at (s) {5'};
        \node[anchor = south west] at (e) {3'};

    \end{tikzpicture}

    \begin{tikzpicture}[overlay, remember picture, shorten >= 7, shorten <= 7, thick]
        \draw[->] (frame cs:0.3,0.0) to[bend right] node[midway,anchor = west] {$\times n$} (frame cs:0.3,-0.5) ;
        \draw[->] (frame cs:0.3,0.5) to[bend right] node[midway,anchor = west] {$\times 2$} (frame cs:0.3,0) ;
    \end{tikzpicture}


\end{frame}


\section{Plan, kinda}
\begin{frame}{General plan}
    \begin{enumerate}
        \item Analyze an ODE system for baseline comaparison
        \item Reformulate the system into stochastic system
        \item Fokker-Planck
        \item Compare against data from building 28
            \begin{enumerate}
                \item Data is not yet made due to experimental blunders
            \end{enumerate}
    \end{enumerate}
\end{frame}

\begin{frame}{ODE system}

\begin{tikzpicture}[remember picture, overlay, thick, shorten >=2pt]
        \node[draw=black, circle] (c1) at (frame cs:0.4, 0.2) {\(c_1\)};
        \node[draw=black, circle] (c2) at (frame cs:0.8, 0.2) {\(c_2\)};

        \node[draw=black] (m) at (frame cs:0.4, -0.4) {\(m\)};


        \draw[->] (c2) [bend left]to node[midway, below] {\(w_2\)} (c1);
        \draw[->] (c1) [bend left]to node[midway, above] {\(w_1\)} (c2);

        \draw[->] (c1)  [loop left]to node[midway, left]  {\(k_1\)} (c1);
        \draw[->] (c2) [loop right]to node[midway, right] {\(k_2\)} (c2);

        \draw[-|] (m) to node[pos=0.35] (mm) {} (c1);
        \draw[-|] (c2) [bend left, out=65, in=155]to node[midway,below] {\(q_m\)} (mm);

        \node[left] at (mm) {\(q_1\)};
\end{tikzpicture}


\begin{columns}
    \column{0.7\framewidth}
    \vspace {1em}
    \begin{itemize}
        \item Simple system of coupled odes
        \item Based off of a compartment model
    \end{itemize}
    \vspace {1em}

    \begin{system} 
        \ode{c_1} &= k_1 c_1\left({n_1 - c_t \over n_1}\right) - w_1 c_1 + w_2 c_2 - q_1 m c_1 \nonumber \\
        \ode{c_2} &= k_2 c_2\left({n_2 - c_t \over n_2}\right) - w_2 c_2 + w_1 c_1\nonumber\\
        \ode{m}   &= m_0 - q_m m c_2 \nonumber
    \end{system} 

    \column{0.3\framewidth}
\end{columns}

    
\end{frame}


\begin{frame}{Fokker-Planck}

    \begin{itemize}
        \item The \emph{Fokker-Planck equation}
            \begin{equation*}
                \pde*[t] P\of{x,t} = 
                -\pde*[x] \br{a^{(1)}_t \of x \cdot P\of{x,t}} + 
                {1 \over 2} \pde*[x]^2 \br{a^{(2)}_t \of x\cdot P\of{x,t}}
            \end{equation*}

        \item Interestingly, all fokker-planck equations must fufill the markov property
        \item Probability distribution as a partial differential equation
        \item Can be defined analytically or numerically
    \end{itemize}
    
\end{frame}

\begin{frame}{Markov Chains}

    \begin{columns}
        \column{0.5\framewidth}
        \begin{itemize}
            \item The probability of transitioning from any state to any other 
                state must not be influence by previous states
            \item Time discrete lends well to cells
            \item The steady state is defined as \(\vec{\pi} M = \vec{\pi}\)
        \end{itemize}
        \column{0.5\framewidth}
        
    \end{columns}

\begin{tikzpicture}[thick,overlay, remember picture]
    \node[draw=black, circle, minimum width = 1cm] (c1) at (frame cs:0.3,0.2) {$x$};
    \node[draw=black, circle, minimum width = 1cm] (c2) at (frame cs:0.6,-0.2) {$y$};

    \begin{scope}[shorten >= 2pt]
        \draw[->] (c1) to[bend left] node[midway, anchor=south west] {\(p_1\)} (c2);
        \draw[->] (c2) to[bend left] node[midway, anchor=north east] {\(p_2\)} (c1);

        \draw[->] (c1) [loop left]to  node[midway] {\(1-p_1\)} (c1);
        \draw[->] (c2) [loop right]to node[midway] {\(1-p_2\)} (c2);
    \end{scope}

    \node at (frame cs:0.4,-0.6) { \(M = \left[
                    \begin{matrix}
                        P\of{x\to x} & P\of{x\to y} \\
                        P\of{y\to x} & P\of{y\to y}
                    \end{matrix}
                \right]
                = 
                \left[
                    \begin{matrix}
                        1-p_1 & p_1 \\ p_2 & 1-p_2
                    \end{matrix}
                \right]\) };
\end{tikzpicture}
    
\end{frame}

\section{Analytical Examples}

\def\cfp#1{c_#1^*}
\begin{frame}{Analytical Solution}

    \begin{columns}[T]
        \column{0.5\framewidth}
    \begin{alertblock}{Notation}
        We denote \(\omega_{i} := k_i-w_i\) and
        \(\k_{i} := \rfrac{k_i}{n_i}\) for all \(i\in \left\{1,2\right\}\).
    \end{alertblock}

        Solving \(\ode* m = 0\) under the assumption that \(m_0=0\),
        \begin{equation*}
            \ode* m = -q_m m c_2 = 0 \implies 
            \begin{cases}
                m\of t = 0 \\
                c_2\of t = 0
            \end{cases}
        \end{equation*}
        as \(c_2 \neq 0\), \(m=0\) for the when \(m_0=0\).

        \column{0.5\framewidth}
        \begin{system}
            \ode* c_1 &= c_1 \pa{\omega_{1} - \k_{1} c_1} + c_2\pa{w_2 - \k_{1} c_1}\nonumber\\
            \ode* c_2 &= c_1\pa{w_1 - \k_{2} c_2} + c_2 \pa{\omega_{2} - \k_{2} c_2}\nonumber
        \end{system}

        
helping us in order to find our nullclines.
\begin{align*}
    c_2 &= -{ c_1 \pa{ \omega_{1} - \k_{1} c_1} \over w_2 - \k_{1} c_1 }
    \intertext{Similarly, solving for \(c_1\) from \(\ode* c_2\) results in} 
    -{c_2\pa{\omega_{2} - \k_{2} c_2} \over w_1 - \k_{2}c_2} &= c_1
\end{align*}
    \end{columns}

\end{frame}

\begin{frame}{Analytical Solution}



        \begin{itemize}
            \item Any solution must have a balance

            \item We can define a level set in the phase space,
                \begin{align*} 
                    c_T
            &=
            \left\{
                \pa{c_1, c_2} \Big| {c_1 \over n_1}  + {c_2 \over n_2} = 1
            \right\} \\
            \implies
                    c_2 &= n_2\pa{1 - {c_1 \over n_1}}
                \end{align*}

        \end{itemize}

        \begin{itemize}
            \item Therefor, we find the point of intersection
                \begin{align*}
                    n_2\pa{1 - {c_1 \over n_1}}  &= 
                    -{c_1\pa{\omega_{1} - \k_{1} c_1} \over w_2 - \k_{1}c_1} 
                \end{align*}
        \end{itemize}


\end{frame}

\begin{frame}{Analytical Solution}
    Which simplifies to solving the polynomial
    \begin{align*} 
        \k_1\pa{{n_2 \over n_1}-1} c_1^2 
        - \pa{\omega_{1} - n_2\pa{{w_2 \over n_1} + \k_1}} c_1
        + n_2 w_2
            &= 0 & c_1\in ]w_2/\k_{1}, \omega_1/\k_1[
            \intertext{by symmetry,} 
            \k_2\pa{{n_1 \over n_2}-1} c_2^2 
            - \pa{\omega_{2} - n_1\pa{{w_1 \over n_2} + \k_2}} c_2
            + n_1 w_1
            &= 0 & c_2\in ]w_1/\k_{2}, \omega_2/\k_2[ 
    \end{align*}

    Resulting in our analytically defined fixed points of \(\vec{p}_1:=\pa{\cfp 1, \cfp 2, 0}\),
\end{frame}

\begin{frame}{Analytical Solution Examples}
    \begin{tikzpicture}[remember picture, overlay]
        \node (image1) at (frame cs:-0.5,0) {
                \includegraphics[width=0.55\framewidth]
                {ode_solution_m00.0_n155.0_n285.0_w10.015_w20.14.pdf}
            };
        \node (image2) at (frame cs:0.5,0) {
                \includegraphics[width=0.55\framewidth]
                {phaseplane_m00.0_n155.0_n285.0_w10.015_w20.14.pdf}
            };
    \end{tikzpicture}
\end{frame}

\begin{frame}{Analytical Solution Examples}
    \begin{tikzpicture}[remember picture, overlay]
        \node (image1) at (frame cs:-0.5,0) {
                \includegraphics[width=0.55\framewidth]
                {ode_solution_m00.0_n1100.0_n290.0_w10.015_w20.015.pdf}
            };
        \node (image2) at (frame cs:0.5,0) {
                \includegraphics[width=0.55\framewidth]
                {phaseplane_m00.0_n1100.0_n290.0_w10.015_w20.015.pdf}
            };
    \end{tikzpicture}
\end{frame}



\begin{frame}{Basic Markov Chain Solution}
    Finding the stationary distribution of a markov chain in the simplest case is done as,
\begin{align*}
    \vec{\pi} M = \vec{\pi} 
    &\implies \pa{\vec{\pi}M}^{\mathrm{T}} = \vec{\pi}^{\mathrm{T}} \\
    % &\implies \pa{M^{\mathrm{T}}\vec{\pi}^{\mathrm{T}}-\vec{\pi}^{\mathrm{T}}}
    % =\pa{M^{\mathrm{T}}-I} \vec{\pi}^{\mathrm{T}}= 0 \\
    &\implies \left[
        \begin{matrix}
            1-p_1 - 1 & p_2 \\
            p_1 & 1-p_2 -1
        \end{matrix}
    \right]
    \left[
        \begin{matrix}
            x \\ y
        \end{matrix}
    \right]
    =
    \left[
        \begin{matrix}
            0 \\ 0
        \end{matrix}
    \right]\\
    &\implies \left[
        \begin{matrix}
            -p_1 & p_2 \\
            p_1 & -p_2
        \end{matrix}
    \right]
    \left[
        \begin{matrix}
            x \\ y
        \end{matrix}
    \right]
    =
    \left[
        \begin{matrix}
            -p_1 x + p_2 y \\
             p_1 x - p_2 y
        \end{matrix}
    \right]
    =
    \left[
        \begin{matrix}
            0 \\ 0
        \end{matrix}
    \right]
\end{align*}

Which we solve and normalize to get the state vector, \(\vec{\pi}\),
with respect to the simple system
\begin{equation*}
    \vec{\pi}
    =
    \left[
        \begin{matrix}
            \cfrac{p_2}{p_2 + p_2} & \cfrac{p_1}{p_1 + p_2}
        \end{matrix}
    \right] 
\end{equation*}
\end{frame}

\begin{frame}{Basic Markov Chain Results}

    \begin{tikzpicture}[remember picture, overlay]
        \node (image1) at (frame cs:-0.5,0) {
                \includegraphics[width=0.55\framewidth]
                {chain-0.7-0.4.pdf}
            };
        \node (image2) at (frame cs:0.5,0) {
                \includegraphics[width=0.55\framewidth]
                {chain-0.015-0.035.pdf}
                %{phaseplane_m00.0_n1100.0_n290.0_w10.015_w20.015.pdf}
            };
    \end{tikzpicture}

\end{frame}

\section{RQ}
\begin{frame}{Intro}

    So the likely research question is
    \begin{block}{RQ}
        How applicable is the Fokker-Planck equation to understanding the dynamics
        of sub populations of hetero-resistent bacteria in a colony.
    \end{block}

\end{frame}


\section{End}
\begin{frame}{Questions?}
    \large Questions?
\end{frame}


\end{document}
