\documentclass[aspectratio=169,  notheorems, sOuRcEs]{RUCPresentation}

%\usepackage{pgfforeach}
% \usepackage{foreach}

\newcounter{chapter}


\usepackage{tikz}
\usetikzlibrary{calc}
\usetikzlibrary{angles}
\usetikzlibrary{positioning}

\usetikzlibrary{arrows.meta}
\usetikzlibrary{shapes.geometric}
\usetikzlibrary{overlay-beamer-styles}

\ProvideDocumentCommand\pgfmathsetnewmacro{ m m }{ \newcommand*{#1}{} \pgfmathsetmacro{#1}{#2} }
\ProvideDocumentCommand\pgfmathsetnewlengthmacro{ m m }{ \newcommand*{#1}{} \pgfmathsetlengthmacro{#1}{#2} }

\tikzset{>=Stealth} % for good looking LaTeX arrow head
% \tikzset{overlay, remember picture}


\usepackage{myMathQA}

\usepackage{graphicx}
\usepackage{siunitx}
\usepackage{subcaption}

\usepackage{hyperref}

% \usepackage[style=authoryear]{biblatex}
\usepackage{biblatex}
\addbibresource{mscbib.bib}

\DeclareSourcemap{
    \maps[datatype=bibtex]{
        \map[overwrite]{
            \step[fieldset=note, null]
        }
        \map[overwrite]{
            \step[fieldset=ppn_gvk, null]
        }
        \map[overwrite]{
            \step[fieldset=file, null]
        }
        \map[overwrite]{
            \step[fieldset=doi, null]
        }
        \map[overwrite]{
            \step[fieldset=issn, null]
        }
        \map[overwrite]{
            \step[fieldset=isbn, null]
        }
        \map[overwrite]{
            \step[fieldset=pagetotal, null]
        }
    }
}

% \DeclareCiteCommand{\beamerfootcite}
%   {}
%   {\footnote[frame]{\printbibliography[keys=\thefield{entrykey}]}}
%   {\multicitedelim}
%   {\usebibmacro{postnote}}

\ProvideDocumentCommand{\beamerfootcite}{ m }{\footnote[frame]{\ \fullcite{#1}}}
\renewcommand*{\nameyeardelim}{\addcomma\space} % cleaner authoryear style

% \usepackage{animate}



\def\k{\raisebox{1ex}{\rotatebox{180}{k}}}
% \newsavebox\tempboxbeamer

\usetheme{RUCTHEME}
% \usetheme{Warsaw}

\makeatletter
\renewcommand\@makefnmark{\hbox{\@textsuperscript{\usebeamercolor[fg]{footnote mark}\usebeamerfont*{footnote mark}[\@thefnmark]}}}
% \renewcommand\@makefntext[1]{{\usebeamercolor[fg]{footnote mark}\usebeamerfont*{footnote mark}[\@thefnmark]}\usebeamerfont*{footnote} #1}
\makeatother

\hypersetup{colorlinks = false, citecolor=TextColor}
\AtBeginBibliography{\small\color{TextColor}}


\def\dumbmatrixtwobytwo[#1,#2]{%
    \count6=#1\relax \count7=#2\relax%
    \count2=#1\relax \count3=#2\relax \count4=#1\relax \count5=#2\relax%
    \multiply\count2 by \count6\relax \multiply\count4 by \count7\relax%
    \multiply\count3 by \count6\relax \multiply\count5 by \count7\relax%
    \ensuremath{%
        \left[%
            \begin{matrix}%
                \number\count2 & \number\count4 \\%
                \number\count3 & \number\count5%
            \end{matrix}%
        \right]%
    }
}


\def\dumbmatrixthreebythree[#1,#2,#3]{%
    \count10=#1\relax \count11=#2\relax \count12=#3\relax
    \count1=#1\relax \count4=#1\relax \count7=#1\relax%
    \count2=#2\relax \count5=#2\relax \count8=#2\relax%
    \count3=#3\relax \count6=#3\relax \count9=#3\relax%
    \multiply\count1\count10\relax \multiply\count4\count11\relax \multiply\count7\count12\relax%%
    \multiply\count2\count10\relax \multiply\count5\count11\relax \multiply\count8\count12\relax%%
    \multiply\count3\count10\relax \multiply\count6\count11\relax \multiply\count9\count12\relax%%
    \ensuremath{%
        \left[%
            \begin{matrix}%
                \number\count1 & \number\count4 & \number\count7 \\%
                \number\count2 & \number\count5 & \number\count8 \\%
                \number\count3 & \number\count6 & \number\count9%
            \end{matrix}%
        \right]%
    }
}

\usepackage{multirow}
\newcounter{reaction}
\ProvideDocumentCommand\reactionterm{ o m m m m }{% --> Function for reaction lines
    \stepcounter{reaction} %
    \IfNoValueTF{#1}
    { \arabic{reaction} & #3 & #4 & \(t^+_{\arabic{reaction}}\of{\vec{x}}\) & #2 & \(#5\) \\ \hline }
    { \multirow{2}*{\arabic{reaction}} & #3 & #4 & \(t^+_{\arabic{reaction}}\of{\vec{x}}\) & #2 &  \multirow{2}*{\(#5\)} \\
                                       & #4 & #3 & \(t^-_{\arabic{reaction}}\of{\vec{x}}\) & #1 &                             \\ \hline }
}

\ProvideDocumentEnvironment{reactiontable}{}%
{ \begin{center} \begin{tabular}{c|r@{\ \(\to\)\ }lr@{\ \(=\)\ }lc} %
    \(I\) & \multicolumn{2}{c}{`Reaction'} %
          & \multicolumn{2}{c}{`Rate Law'} %
          & \(\vec{r}_I^{\mathrm{T}}\) \\\hline }%
{ \end{tabular} \end{center} \setcounter{reaction}{0} }


% Defining a new coordinate system for the page:
%
% --------------------------
% |(-1,1)    (0,1)    (1,1)|
% |                        |
% |(-1,0)    (0,0)    (1,0)|
% |                        |
% |(-1,-1)   (0,-1)  (1,-1)|
% --------------------------


% For beamer title
\newcommand\makebeamertitle{\frame{\maketitle}}%

% For sections?
\graphicspath{{./images/}{../Thesis/images/}{../Thesis/images/phase/}{../Thesis/images/ode/}{../Thesis/images/fpe/}}


\def\cfp#1{c_#1^*}


\title{%
    Some Stuff Related to my Thesis
}
\subtitle{Presentation For The MSc Thesis Exam}

\author{by Anakin}

%\mode<presentation>


\begin{document}


\makebeamertitle


% \begin{frame}{Outline}
%
%     \tableofcontents
%
% \end{frame}



\section{Introduction}
\begin{frame}{Introduction}

    \begin{enumerate}
        \item Welcome/Introduction
        \item Brief reminder of the themes/goals
        \item Missing Details
            \begin{enumerate}
                \item Martingales
                \item Birth-Death Processes
                \item How it applies with a 1 dimensional version of the model
                    I used.
            \end{enumerate}
    \end{enumerate}

\end{frame}


% TODO:
\section{ Motivation }
\begin{frame}{ Motivation }

    \begin{columns}

        \column{0.5\framewidth}

        \begin{enumerate}
            \item Statistics are weird and powerful
            \item Biology is weird and incredibly adaptive
            \item If the physicists could use statistics to explain the bizzare
                nature of quantum tomfoolery
            \item Adaptation and power can be considered the same
            \item Is statistics and biology the same?
        \end{enumerate}

        \column{0.5\framewidth}

    \end{columns}

\end{frame}

\begin{frame}{The Research Question}

    \begin{enumerate}
        \item There was a lot of indecision about the direction to take things
        \item
    \end{enumerate}

    \begin{alertblock}{The Reseach Question:}
            ``How can stochastic gene copy number dynamics underlying bacterial
            heteroresistance be accurately captured using a Fokker-Planck formulation,
            and what insights does this stochastic framework provide beyond deterministic ODE models.''
    \end{alertblock}

\end{frame}


\begin{frame}{The process}

    \begin{itemize}
        \item I had to cover a lot of novel ground
        \item
    \end{itemize}

\end{frame}


\section{Gene stuff}
\begin{frame}{ Recapping the Model }


    \begin{columns}

        \column{0.5\framewidth}

        \column{0.5\framewidth}

    \end{columns}

    \tikz[overlay, remember picture] \node at (frame cs:0.35,0) {%
        \begin{tikzpicture}
            \pgfmathsetnewmacro\genelength{35}
            \pgfmathsetnewmacro\geneheight{2}

            \pgfmathsetnewmacro\translength{10}
            \pgfmathsetnewmacro\transheight{2}

            \pgfmathsetnewmacro\reactionthickness{1}
            \pgfmathsetnewmacro\dnawidth{4}

            \pgfmathsetnewmacro\shortendist{6}
            \pgfmathsetnewmacro\bendstrength{1.05} % √(1²+(0.2-0.75)²)

            \pgfmathsetnewmacro\geneshift{5}

            \providecommand*\genecolor{red!80!black}
            \providecommand*\transcolor{green!80!black}

            \useasboundingbox (-4,-1.25) rectangle (5.5,1.35);
            \coordinate (O) at (0,0);

            \node[circle] (west) at (-3.5, 0) {};
            \node[circle] (east) at ( 4.5, 0) {};

            \node[anchor=south east] at (west) {\(5'\)};
            \node[anchor=south west] at (east) {\(3'\)};

            \draw[line width = \dnawidth] (west.center) to (east.center);

            \draw[line width = \dnawidth] (east.north) to (east.south);
            \draw[line width = \dnawidth] (west.north) to (west.south);


            \coordinate (g1) at (-2,0); \coordinate (g2) at ( 0,0);
            \coordinate (g3) at ( 3,0);

            \coordinate (t1) at (-3,0); \coordinate (t2) at (-1,0);
            \coordinate (t3) at ( 1,0);
            \coordinate (t4) at ( 2,0); \coordinate (t5) at ( 4,0);

            \draw[line width = \dnawidth+1, BackgroundColor, dashed] (t3) -- (t4);
            \draw[line width = \dnawidth+1, BackgroundColor] ([xshift=0.37cm]t3) to ([xshift=-0.43cm]t4);

            \foreach \POINT in {1,2,3}{
                \node[rectangle, fill=\genecolor, draw=\genecolor, minimum width=\genelength, minimum height=\geneheight] at (g\POINT) {};
            }

            \foreach \POINT in {1,2,3,4,5}{
                \node[rectangle, fill=\transcolor, draw=\transcolor, minimum width=\translength, minimum height=\transheight] at (t\POINT) {};
            }

            \node[above, yshift=5] (x1) at (g1) {\(X\)};
            \node[above, yshift=5] (x2) at (g2) {\(X\)};
            \node[above, yshift=5] (x3) at (g3) {\(X\)};

            \node  (B) at ( 4.75, 0.85) {\(B\)};
            \node (B2) at ( 0.50, 0.85) {\(B\)};

            \node  (N) at ( 2.00,-0.75) {\(N\)};
            \node (N2) at (-2.00, 1.00) {\(N\)};
            \node (N3) at ( 2.75, 1.00) {\(N\)};

            \node  (y) at ( 4.00,-0.75) {\(Y\)};
            \node (y2) at ( 1.50, 0.30) {\(Y\)};
            \node (y3) at ( 0.75,-0.75) {\(Y\)};

            \begin{scope}[line width=\reactionthickness, shorten <=\shortendist, shorten >=\shortendist]
                \draw[->] ([yshift=\geneshift]g1)
                    to[bend left] node[midway, sloped] (k1path) {}
                    ([yshift=\geneshift]g2);
                \draw[->] ([yshift=-\geneshift]g2)
                    to[bend left] node[midway] (k-1path) {}
                    ([yshift=-\geneshift]g1);
            \end{scope}

            \node[above] at (k1path) {\(k_1\)};
            \node[below] at (k-1path) {\(k_{-1}\)};

            \draw[line width=\reactionthickness, shorten >=6pt]
                (N2) to[bend right] ([yshift=2.3]k1path.center);

            % \draw[->,line width=\reactionthickness, shorten <=5.5pt]
            %     ([yshift=-2.6]k-1path.center) to[bend right] (N3);



            \begin{scope}[line width = \reactionthickness]
                \draw[shorten >=5pt, shorten <=4pt ]
                    (B.center) -- (x3.center)
                    node[pos=0.5, sloped] (midbx) {\(\bigtimes\)};
                \draw[shorten >=5pt, shorten <=4pt ]
                    (B2.center) -- (y2.center)
                    node[pos=0.5, sloped] (midby) {\(\bigtimes\)};

                \draw[->] (midbx.center) to[in=0, out=105, looseness=1.5] (N3);
                \draw[->] (midbx.center) to[in=160, out=105, looseness=3] (B);

                \draw[->] (midby.center) to[in=180, out=71, looseness=1.5] (N3);

                % \draw[->] (midxb.center) -- ++(105:0.5);
                % \draw[->] (midxb.center) to[bend right] (N2.east);

                \draw[->] (N) .. controls ++( 45:\bendstrength) and ++(135:\bendstrength) .. (y)
                    node[midway, below] {\(k_2\)};
                \draw[->] (y3) -- (N) node[midway, above] {\(k_5\)};
            \end{scope}

            \node[anchor=north west] at (midbx) {\(k_3\)};
            \node[anchor=west] at ([yshift=2,xshift=1]midby) {\(k_4\)};


        \end{tikzpicture}
    };


\end{frame}


% \begin{frame}{ Proof }
%
%     \begin{columns}
%         \column{ 0.5\framewidth}
%
%         \begin{description}
%             \item[S:] Statistics
%             \item[B:] Biology
%             \item[R:]
%         \end{description}
%
%         | \hfill |
%
%         \column{ 0.5\framewidth}
%
%         \begin{description}
%             \item \(P \implies Q\)
%             \item \(Q \implies R\)
%             \item \(P \implies Q, Q \implies R \vdash P \implies R\)
%         \end{description}
%
%         | \hfill |
%
%     \end{columns}
%
% \end{frame}

\begin{frame}{ Birth Death Processes }

    \begin{columns}%[C]

        \column{0.5\framewidth}

        \begin{enumerate}
            \item A discretized form of random walk
            \item Stationary form takes the appearance
                \begin{equation}
                    0 = J\of{x+1} - J\of x
                \end{equation}
        \end{enumerate}

        \column{0.5\framewidth}

    \end{columns}


\end{frame}

% \item Distinguished by not inherently


% TODO:
\section{My Model in 1-Dimension}
\begin{frame}{Simplest Novel Model}

    \begin{enumerate}
        \item It somehow never crossed my mind to make my model in one dimension
        \item So we present it
    \end{enumerate}

    \begin{columns}%[C]

        \column{0.5\framewidth}

        \begin{reactiontable}
            \reactionterm[\(k_{-1} x^2\)]{\(k_1 n x\)}{\(X+N\)}{\(2 X\)}{ 1 }
            \reactionterm{\(k_3 b x\)}{\(X+B\)}{\(B+N\)}{ -1 }
        \end{reactiontable}

        \column<2->{0.4\framewidth}

        \begin{align*}
            A &= k_{-1} x^2 + \pa{ k_3 b - k_1 n } x
            \\
            B &= {1 \over \Omega} \pa{k_{-1} x^2 + \pa{ k_1 n + k_3 b } x}
        \end{align*}

    \end{columns}



\end{frame}

\begin{frame}{ Analytical Distribution }

    \begin{columns}

        \column{0.5\framewidth}

        \column{0.5\framewidth}

    \end{columns}

    \begin{tikzpicture}[overlay, remember picture]
        \node at (frame cs:0.5, 0.0) {
                \includegraphics[width=0.5\textwidth]
                {Novel1VarData.pdf}
            };
    \end{tikzpicture}

\end{frame}


% TODO:
\section{Revisiting the Proof}
\begin{frame}{The Proposition}

    \begin{proposition}[name={Stochastic Stability Decay}] \label{prp.ssd}

        %which we can represent with \gls{fpe}
        Assume \(M\) is a manifold denoting a state space with an unstable origin,
        whose {evolution rule} is \(\varphi:\R\times M\to M\),
        \(t\in\R\), and that \(N\subset M\) is a submanifold that includes the origin.

        % with an attracting point \(\vec{x}^*\in N\backslash\{\vec{0}\}\).
        Additionally assume we have system of SDEs,
        which involves an isolated autocatalytic process,
        \begin{equation*}
            \delta \vec{x}\of t
            = \vec{A}\br{\vec{x}\of t, t}
            + \sqrt{\bfm B\br{\vec{x}\of t, t}} \ \delta{\vec{W}\of{t\st \vec{x}}}
        \end{equation*}
        Such that the drift term \(\vec{A}:N\to\R^n\) is the vector field
        associated with the flow \(\varphi_t\of N\).
        % homeomorphic to \(N\).

        If \(\varphi_t\of{N}\) can be partitioned into a trapping region,
        \(N\backslash\{\vec{0}\}\), and the unstable point \(\{\vec{0}\}\).
        Then given a large enough time interval,
        the stochastic system will inevitably decay from stability.

    \end{proposition}

\end{frame}

\begin{frame}{ Elaboration }

    \begin{columns}

        \column{0.5\framewidth}

        \column{0.5\framewidth}

    \end{columns}

\end{frame}


\begin{frame}{Applied to the 1D Model}

    \begin{enumerate}
        \item From the Proposition, we see that 1-var will decay
        \item The behavior is then of course easily carried to the higher
            dimensional models
        \item As established
    \end{enumerate}

\end{frame}


% \begin{frame}{The Corollary}
%
%     \begin{corollary} \label{con:timetodestable}
%
%         Assuming that we have a system that fufills the properties of a stochastic
%         stability decay.
%         And assuming that \(Z\) is a standard normal random variable.
%         Then, the median interval of time, \(t-s\), \(s<t\in[0,\infty[\),
%         needed to hit the origin, will be dependent on the distance
%         \(\mu\of{0, x^*}= L\) and the diffusion term, such that
%         \begin{equation*}
%             t-s = \Delta t = \pa{
%                 {L \over z_\alpha \sqrt{|\nabla \cdot \bfm B\of{\vec{x}^*}|}}
%             }^2
%         \end{equation*}
%         will give a \(100\pa{1-\alpha}\) confidence interval for the
%         system to decay to zero.
%
%     \end{corollary}
%
% \end{frame}


% \begin{frame}{Does this even make sense as a claim?}
%
%     \begin{align*}
%         \nabla\cdot \mat{B} &= \nabla \cdot \bmatform[ -k_{-1} x + k_1 y,  k_{-1} x - k_1 y; k_{-1} x - k_1 y, -k_{-1} x + k_1 y ]
%          = \bmatform[ 0, 0]
%     \end{align*}
%
%     \begin{align*}
%         \nabla\cdot \mat{B} &= \nabla \cdot \bmatform[
%         \pa{n +  c_1 + c_2} k_1 c_1 +  w_1 c_1 + w_2 c_2 ,
%         - \pa{ w_1 c_1 + w_2 c_2  };
%         - \pa{ w_1 c_1 + w_2 c_2  },
%         \pa{n +  c_1  + c_2} k_2 c_2 + w_1 c_1 + w_2 c_2
%         ] \\
%         &= \bmatform[
%         \pa{n_1 + 2c_1 +  c_2}k_1,
%         \pa{n_2 +  c_1 + 2c_2}k_2
%         ] \\
%        \implies |\nabla\cdot \mat{B}\of{\vec{x}^*}| &= \sqrt{
%            (\pa{n_1 + 2c_1 +  c_2}k_1)^2 + (\pa{n_2 +  c_1 + 2c_2}k_2)^2
%        }
%     \end{align*}
%
% \end{frame}

% % TODO:
% \section{Extension of the Framework}
% \begin{frame}{frame title}
%     frame contents
% \end{frame}


\section{ Discussion Points }
\begin{frame}{Fuck}
    frame contents
\end{frame}


\begin{frame}{Fuck}
    frame contents
\end{frame}


\begin{frame}{Bibliography}



\end{frame}





\appendix

\begin{frame}{ Flux Derivation }



\end{frame}


\end{document}
