\documentclass[aspectratio=169]{RUCPresentation}

%\usepackage{pgfforeach}
% \usepackage{foreach}

\usepackage{MyMath}
% \usepackage{animate}

\usepackage{siunitx}

\usepackage{tikz}
\usetikzlibrary{shapes, arrows, positioning, calc}
\usetikzlibrary{overlay-beamer-styles}


\def\k{\raisebox{1ex}{\rotatebox{180}{k}}}
% \newsavebox\tempboxbeamer

% \usetheme{Warsaw}
\usetheme{RUCTHEME}


% Defining a new coordinate system for the page:
%
% --------------------------
% |(-1,1)    (0,1)    (1,1)|
% |                        |
% |(-1,0)    (0,0)    (1,0)|
% |                        |
% |(-1,-1)   (0,-1)  (1,-1)|
% --------------------------

% For beamer title
\newcommand\makebeamertitle{\frame{\maketitle}}%

% For sections?
\graphicspath{{./images/}{../Thesis/images/phase/}{../Thesis/images/ode/}{../Thesis/images/markov/}}


\title{%
    Applying Fokker-Planck to analyze bacterial populations exhibiting hetero-resistence properties%
}
\title{%
    Understanding statistical distributions hetero-resistant bacterial populations through%
    fokker-planck or generalized statistical manifold%
}
\subtitle{  Mathematical\ \ Bioscience MSc }
% \newline supervisor: Morten Anderson}

\author{Anakin}

%\mode<presentation>

\begin{document}


\makebeamertitle


\section{Introduction}

\begin{frame}{The Research Question}
\end{frame}

\section{Motivation}


\begin{frame}{Heteroresistence}
\end{frame}

\subsection{Copy Numbers}

\begin{frame}{Gene Copy Numbers}
\end{frame}


\begin{frame}{Fitness Doesn't Matter?}
\end{frame}



\section{Plan, kinda}

\begin{frame}{General plan}
\end{frame}

\begin{frame}{ODE system}
\end{frame}


\begin{frame}{Fokker-Planck}

\end{frame}

\begin{frame}{Markov Chains}

\end{frame}

\section{The Ordinary Differential System}

\begin{frame}{Two Species Compartment}

\end{frame}

\def\cfp#1{c_#1^*}
\begin{frame}{Analytical Solution}

    \begin{columns}[T]
        \column{0.5\framewidth}
    \begin{alertblock}{Notation}
        We denote \(\omega_{i} := k_i-w_i\) and
        \(\k_{i} := \rfrac{k_i}{n_i}\) for all \(i\in \left\{1,2\right\}\).
    \end{alertblock}

        Solving \(\ode* m = 0\) under the assumption that \(m_0=0\),
        \begin{equation*}
            \ode* m = -q_m m c_2 = 0 \implies
            \begin{cases}
                m\of t = 0 \\
                c_2\of t = 0
            \end{cases}
        \end{equation*}
        as \(c_2 \neq 0\), \(m=0\) for the when \(m_0=0\).

        \column{0.5\framewidth}
        \begin{system}
            \ode* c_1 &= c_1 \pa{\omega_{1} - \k_{1} c_1} + c_2\pa{w_2 - \k_{1} c_1}\nonumber\\
            \ode* c_2 &= c_1\pa{w_1 - \k_{2} c_2} + c_2 \pa{\omega_{2} - \k_{2} c_2}\nonumber
        \end{system}


helping us in order to find our nullclines.
\begin{align*}
    c_2 &= -{ c_1 \pa{ \omega_{1} - \k_{1} c_1} \over w_2 - \k_{1} c_1 }
    \intertext{Similarly, solving for \(c_1\) from \(\ode* c_2\) results in}
    -{c_2\pa{\omega_{2} - \k_{2} c_2} \over w_1 - \k_{2}c_2} &= c_1
\end{align*}
    \end{columns}

\end{frame}



\begin{frame}{Analytical Solution}
\end{frame}


\begin{frame}{Analytical Solution}
\end{frame}

\begin{frame}{Analytical Solution Examples}
\end{frame}

\begin{frame}{Analytical Solution Examples}
\end{frame}





\section{End}
\begin{frame}{Questions?}
    \large Questions?
\end{frame}


\end{document}
